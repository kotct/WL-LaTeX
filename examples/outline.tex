\documentclass[12pt,letterpaper]{article}

\usepackage{mla}
\usepackage{outlines}

\renewcommand{\theenumi}{\Roman{enumi}. }
\renewcommand{\labelenumi}{\theenumi}

\renewcommand{\theenumii}{\Alph{enumii}. }
\renewcommand{\labelenumii}{\theenumii}

\renewcommand{\theenumiii}{\arabic{enumiii}. }
\renewcommand{\labelenumiii}{\theenumiii}

\renewcommand{\theenumiv}{\alph{enumiv}. }
\renewcommand{\labelenumiv}{\theenumiv}

\begin{document}

\begin{mla}{Christopher}{Cooper}{Mrs.\ Mills}{English 9 H}{9 September 2012}{Literary Analysis Paper Outline}

\begin{outline}[enumerate]
\1 Introduction
   \2 Arthur C. Clarke
   \2 ``\thinspace`If I Forget Thee, Oh Earth \dots'\thinspace''
   \2 The setting in ``\thinspace`If I Forget Thee, Oh Earth \dots'\thinspace'' by 
   Arthur C. Clarke is an important part of the mood of the story.
\1 Body
   \2 The phosphorescence emmited from the Earth creates an eerie
   feeling that contributes to the story's mood.
      \3 The glow creates a ominous feeling of death, described by Clarke as an 
      ``evil phosphorescence'' or ``the aftermath of Armageddon.''
      \3 You have to look from the dark side of the mood to the 
      dark side of the Earth to see it. Very dark.
      \3 Compared to the bright part of the Earth, it looks much 
      scarier for two reasons:
         \4 The bright side seems inviting, while the glowing dark
         side looks foreboding.
         \4 The bright side looks bright and alive, while the 
         dark side would look peaceful, but instead looks like chaos.
   \2 The lack of humans other than Marvins's father 
   creates a sense of solitude.
      \3 Other people would simply be distracting in the story, 
      detracting from the mood.
      \3 This solitude puts the focus solely on the Earth.
      \3 When ``Father'' tells Marvin the story about Earth, 
      instead of distracting us from Earth, it focuses the story even more.
   \2 The darkness of the surroundings amplifies the importance
   of Earth in the story.
      \3 The Earth is the only souce of light, and seems quite bright.
      \3 ``\dots the valley before them should be in total darkness.
      Yet it was awash with a cold white radiance that came spilling 
      over the crags beneath which they were driving. Then, 
      suddenly, they were out in the open plain, and the source 
      of light lay before them in all its glory.''
      \3 There literally is nothing else to give attention
      to; the Earth is the only thing of interest in sight.
\1 Conclusion
   \2 The setting in ``\thinspace`If I Forget Thee, Oh Earth \dots'\thinspace'' by 
   Arthur C. Clarke is an important part of the mood of the story.
   \2 By focusing on the Earth through the setting or lack thereof, 
   Clarke does a very good job of creating a longing and sad mood.
\end{outline}

%body here

%citations:
%\begin{workscited}
%\bibent
%bibliography entry
%\end{workscited}

\end{mla}

\end{document}
